\documentclass[a4paper,11.5pt]{article}
\usepackage[latin1]{inputenc}
\usepackage[T1]{fontenc}
\usepackage[english]{babel}
\usepackage{graphicx}
\usepackage{amsmath}
\usepackage{amsfonts}
\usepackage{multirow}
\usepackage{booktabs}
\usepackage{bbold}
\usepackage{mathtools}
\usepackage{mathrsfs}
\usepackage{enumitem}
\usepackage{array}

\setlength{\parindent}{0pt}
\DeclarePairedDelimiter{\floor}{\lfloor}{\rfloor}
\DeclarePairedDelimiter{\ceil}{\lceil}{\rceil}

\newcommand{\vt}{\boldsymbol}

\title{Digital Communications - HW3}
\author{Jacopo Pegoraro, Edoardo Vanin}
\date{21/05/2018}

\begin{document}

\maketitle

We have to implement six different versions of the receiver structure in a QPSK modulation scheme. First we present the setup of the transmitter and the channel as given, then we analyze the different configurations one by one and give a brief discussions of the resulting probabilities of symbol error obtained from simulation over different values of the SNR at the channel output, $\Gamma$. 

\section*{Transmitter and Channel}

The system takes a sequence of input symbols $a_k$ at sampling time $T=1$ and applies an upsampling of factor 4, obtaining $a_k'$ at $T/4$. This new sequence is then filtered by $q_c$ as described by the following difference equation:
\begin{equation}
s_c(nT/4) = 0.67 s_c((n-1)T/4) + 0.7424 a_{n-5}
\end{equation}
After the filtering white noise is added. The SNR at the channel output for all the configurations in this first phase is $\Gamma = 10$ dB, so from the following relations we can derive $\sigma_w^2$, the variance of the complex valued Gaussian noise:
\begin{equation}
\Gamma = \frac{M_{s_c}}{N_0\frac{1}{T}} = \frac{\sigma_a^2 E_{q_c}}{\sigma_w^2} \longrightarrow \sigma_w^2 = \frac{\sigma_a^2 E_{q_c}}{\Gamma} = 2\sigma_I^2
\end{equation}
where $\sigma_I^2$ is the variance per component. In addition we can also compute the PSD as $N_0=\sigma_w^2 T_c=\sigma_w^2/4$, because the sampling time $T_c$ at which we add the noise is $T/4$.
In figure \ref{fig:qc} we plot the impulse response and the frequency response of the filter $q_c$.
This implementation of the transmitter is the same for all the following discussion.

\begin{figure}[ht]
	\begin{center}   
		\includegraphics[width=\textwidth]{figs/q_c.png} 
		\includegraphics[width=\textwidth]{figs/Qc.png} 
		\caption{Impulse and frequency response of the filter $q_c$ at $T/4$.}
		\label{fig:qc}
	\end{center}
\end{figure} 

\section*{Point A}

In point A at the receiver we have a matched filter $g_{M}$ (see figure \ref{fig:A_gm}), obtained from $q_c$ as $g_M=q_c^*(t_0-t)$. For simplicity in the last formula we have denoted the filters as is they were defined on continuous time while in the actual simulation they are at $T/4$. 

\begin{figure}[ht]
	\begin{center}   
		\includegraphics[width=\textwidth]{figs/A_gm.png} 
		\caption{Impulse response of the matched filter $g_{M}$ for the receiver in point A.}
		\label{fig:A_gm}
	\end{center}
\end{figure} 

The output of the matched filter is then sampled at $T$ starting from an initial offset called \emph{timing phase} $t_0$. In our case the choice of $t_0$ is made easy by the presence of the matched filter, as we can just choose the value $\bar{t}_0$, multiple of $T/4$, that is the index of the peak of the correlation between $q_c$ and $g_M$, then $t_0$ will be equal to $\bar{t}_0 T/4$. Following this reasoning we chose $\bar{t}_0=17$, equal also to the length of $g_M$ (see figure \ref{fig:A_gm}).

The signal is then passed to a linear equalizer (LE) derived by a particular case of a Decision Feedback Equalizer (DFE) where we only have the feedforward filter $c$ (see point B for the detailed analysis of the DFE). The signal at this point in the receiver system is called $x_k$ and is the result of the convolution of the input sequence $a_k$ with the overall impulse response $h_i = q_c * g_M$ that goes from $-N_1$ to $N_2$. We will call precursors the taps of $h$ that go from $-N1$ to $-1$ and postcursors the taps from $1$ to $N_2$. To obtain the coefficients of $c$ we used the Wiener approach on the input random process and solved the Wiener-Hopf equation $\vt{c}_{opt}=\vt{R}^{-1}\vt{p}$ using the matrix $\vt{R}$ and vector $\vt{p}$ as in equations \ref{eq:wienerR} and \ref{eq:wienerp} with the parameter $M_2$ (the order of the feedback filter) set to $0$ because we have no feedback filter in this case. The free parameters that we have to choose are $M_1$, the order of filter $c$ and $D$ the delay that it will introduce on the input sequence. We carried out this choice looking at the value of the cost function $J_{min}$ (see equation \ref{eq:jmin}) for each combination of the two parameters, preferring low values if possible to avoid increasing the complexity. The best choice in this case was $M_1=5$ and $D=2$.
In figure \ref{fig:A_c} we plot the impulse response $c_i$ at sampling time $T$ as obtained from the Wiener solution. 

\begin{figure}[ht]
	\begin{center}   
		\includegraphics[width=\textwidth]{figs/A_c.png} 
		\caption{Magnitude of the impulse response of filter $c$ in point A.}
		\label{fig:A_c}
	\end{center}
\end{figure}

The aim of filtering with $c$ is to obtain an overall impulse response of the system that satisfies the Nyquist conditions for the absence of ISI at time $T$. This implies that in the ideal case $\psi=h*c$ is a delayed impulse centered on $D$ that is the delay. In our case the $\psi$ obtained is shown in figure \ref{fig:A_psi}. We can see the result is pretty good as all the precursors and postcursors are almost canceled by the equalizer. 


\begin{figure}[ht]
	\begin{center}   
		\includegraphics[width=\textwidth]{figs/A_psi.png} 
		\caption{Magnitude of the impulse response of the system $\psi$ in point A.}
		\label{fig:A_psi}
	\end{center}
\end{figure}

The detected symbols $a_{k-D}$ are chosen using a threshold detector that analyzes the sign of the imaginary and real part of the input complex value. Not that the same detector is also used at point B, C and D.


\section*{Point B}

For point B the system is the same as in point A up to the equalizer, so the choice of $\bar{t}_0=17$ is the same. The matched filter is the same as in point A, see figure \ref{fig:B_gm}. However in this case we equalize with a DFE, that is made of two filters called feedforward and feedback filter denoted by $c$ and $b$. The feedforward filter has the role of equalizing only the precursors of the overall impulse response, while the ISI due to postcursors will be canceled by filter $b$ positioned on a feedback loop between the output of the threshold detector and its input. 

\begin{figure}[ht]
	\begin{center}   
		\includegraphics[width=\textwidth]{figs/B_gm.png} 
		\caption{Impulse response of the matched filter $g_{M}$ for the receiver in point B.}
		\label{fig:B_gm}
	\end{center}
\end{figure}

The computation of the optimal filters $c$ and $b$ is carried out using the Wiener filter approach. The relation between the input random process and the output is:
\begin{equation}
\begin{split}
y_k &= x_{FF,k} + x_{FB,k} \\
&= \sum_{i=0}^{M_1-1}c_ix_{k-i} + \sum_{j=1}^{M_2}b_ja_{k-D-j}
\end{split} 
\end{equation}
where $M_1$ is the order of the feedforward filter, $M_2$ is the order of the feedback filter and $a_{k-D}$ are the already detected past symbols fed back through $b$. Defining postcursors and precursors as in point A, we have that we can apply the Wiener-Hopf equations on the process:
\begin{equation} \label{eq:yk}
y_k = \sum_{i=0}^{M_1-1}c_i \left(x_{k-i}-\sum_{j=1}^{M_2}h_{j+D-i}a_{k-j-D} \right)
\end{equation}

The result can be easily computed as $c_{opt} = \vt{R}^{-1}\vt{p}$ once we find the autocorrelation matrix $\vt{R}$ and the correlation vector $\vt{p}$, expressed as \cite{nevio<3}:

\begin{equation} \label{eq:wienerR}
\mathbf{[R]}_{p,q} = \sigma_a^2 \left( \sum_{j=-N_1}^{N_2}h_jh^*_{j-(p-q)}-\sum_{j=1}^{M_2}h_{j+D-q}h^*_{j+d-p} \right) + r_{\tilde{w}}(p-q)
\end{equation}
\begin{equation} \label{eq:wienerp}
\mathbf{[p]}_p = \sigma_a^2 h^*_{D-p} \quad\quad\quad\quad\quad\quad p = 0,1,\dots,M_1-1
\end{equation}

where for a QPSK scheme $\sigma_a^2=2$ because it is the sum of two orthogonal components each with power $1$. The values of $r_{\tilde{w}}$ are the result of the autocorrelation of the noise after being filtered by $g_M$, so being the noise white we have $r_{\tilde{w}}(n)=N_0r_{g_M}(nT)$. At this point we can define the overall impulse response up to the threshold detector $\psi = h*c_{opt}$ and derive the optimal coefficients for filter $b$ as $b_i=-\psi_{i+D}$ for $i=1,\dots,M_2$.

The value of the cost function $J_{min}$ obtained using these the optimal filters is :
\begin{equation} \label{eq:jmin}
J_{min} = \sigma^2_a \left( 1-\sum_{l=0}^{M_1-1} c_{opt,l}h_{D-l}\right)
\end{equation}

Again the parameters to choose are the order of filter $c$, $M_1$, and the delay introduced $D$. This is because the order of $b$ can be chosen in such a way that all the postcursors are canceled by the feedback: $M_2=N_2+M_1-D-1$, and also the expression of the autocorrelation matrix significantly simplifies. The choice is carried out by selecting the values that minimize the functional $J_{min}$, this time being $M_1=5$ and $D=4$, and consequently $M_2=4$ because $N_2=4$. In figure \ref{fig:B_c}, \ref{fig:B_psi} and \ref{fig:B_b} we plot the resulting filters $c$, $\psi$ and $b$ at sampling time $T$. 

\begin{figure}[ht]
	\begin{center}   
		\includegraphics[width=\textwidth]{figs/B_c.png} 
		\caption{Magnitude of the impulse response of the filter $c$ (feedforward filter) for the receiver in point B.}
		\label{fig:B_c}
	\end{center}
\end{figure}

\begin{figure}[ht]
	\begin{center}   
		\includegraphics[width=\textwidth]{figs/B_psi.png} 
		\caption{Magnitude of the impulse response of the system $\psi$ in point B.}
		\label{fig:B_psi}
	\end{center}
\end{figure}

\begin{figure}[ht]
	\begin{center}   
		\includegraphics[width=\textwidth]{figs/B_b.png} 
		\caption{Magnitude of the impulse response of the filter $b$ (feedback filter) in point B.}
		\label{fig:B_b}
	\end{center}
\end{figure}

\section*{Point C}

In the receiver at point C we use a different type of approach. Before sampling the received signal we use an anti-aliasing filter instead of a matched filter. This is not an optimal solution but can be useful in some situation where the channel is not known or varies in time. Also the downsampling is at $T/2$ instead of $T$ and this gives more degrees of freedom int the equalization and more robustness with respect to the choice of the timing phase. The anti-aliasing filter has to avoid overlapping in frequency when we downsample the signal at the output of the channel. Our signal has frequency content that is periodic of period $4/T$, and has the shape of the Fourier transform of filter $q_c$. Therefore its bandwidth is not limited and our anti-aliasing filter $g_{AA}$ will remove useful information. The sampling at $T/2$ causes the frequency content of the signal to be repeated every $2/T$, causing an overlap. To avoid this, the cutting frequency of $g_{AA}$ has to be around $1/T$. In figure \ref{fig:C_gaa} we show the magnitude of the frequency response $|G_{AA}|$, where we chose the passband at $0.45 \cdot 2/T$ and the stopband at $0.55\cdot 2/T$.

\begin{figure}[ht]
	\begin{center}   
		\includegraphics[width=\textwidth]{figs/GAA.png} 
		\caption{Magnitude of the frequency response of the anti-aliasing filter in point C.}
		\label{fig:C_gaa}
	\end{center}
\end{figure}

\begin{figure}[ht]
	\begin{center}   
		\includegraphics[width=\textwidth]{figs/GM_CD.png} 
		\caption{Magnitude of the frequency response of the matched filter in point C.}
		\label{fig:C_gm}
	\end{center}
\end{figure}

The sampling then has to start after an offset $\bar{t}_0$ equal to the peak of the overall impulse response $q_c*g_{AA}$ at time $T/4$ that in this case was equal to $21$.  
In this kind of configuration we also add a digital matched filter after the sampling that is different from the previous points, now the matched filter is given by flipping and taking the conjugate of the whole impulse response $q_c*g_{AA}$ at sampling time $T/2$:
\begin{equation}
g_M = \left\{ q_c * g_{AA} \right\}^* \left(t_0 + i \frac{T}{2}\right) 
\end{equation}
And the inpulse response is depicted in figure \ref{fig:C_gm}. %Aggiungere la figura per la risposta inpoulsiva del filtro
In this case we use again a DFE. To derive the coefficients of the equalizer. Now the $c$ filter works at $T/2$ so the equations used for the Wiener solution aren't the same used for the points A and B. The equations \ref{eq:yk}, \ref{eq:wienerR} and \ref{eq:wienerp} become
\begin{equation} \label{eq:C_yk}
y_k = \sum_{i=0}^{M_1-1}c_i \left(x_{2k-i}-\sum_{j=1}^{M_2}h_{2(j+D)-i}a_{k-j-D} \right)
\end{equation}
\begin{equation} \label{eq:C_wienerR}
\mathbf{[R]}_{p,q}  = \sigma_a^2 \left( \sum_{n=-\infty}^{\infty}h_{2n-q}h^*_{2n-p}-\sum_{j=1}^{M_2}h_{2(j+D)-q}h^*_{2(j+D)-p} \right) + r_{\tilde{w}}(p-q) \\
\end{equation}
\begin{equation} \label{eq:C_wienerp}
\mathbf{[p]}_p = \sigma_a^2 h^*_{2D-p} \quad\quad\quad\quad\quad\quad p,q = 0,1,\dots,M_1-1
\end{equation}

where for a QPSK scheme $\sigma_a^2=2$ because it is the sum of two orthogonal components each with power $1$. The values of $r_{\tilde{w}}$ are the result of the autocorrelation of the noise after being filtered by $g_{AA}$ and $g_M$, so being the noise white we have $r_{\tilde{w}}(n)=N_0r_{g_M * g_{AA}}(nT/2)$. At this point we can define the overall impulse response up to the threshold detector $\psi = h*c_{opt}$ and derive the optimal coefficients for filter $b$ as $b_i=-\psi_{2(i+D)}$ for $i=1,\dots,M_2$ because filter $b$ works at $T$. 

The value of the cost function $J_{min}$ obtained using these the optimal filters is :
\begin{equation} \label{eq:C_jmin}
J_{min} = \sigma^2_a \left( 1-\sum_{l=0}^{M_1-1} c_{opt,l}h_{2D-l}\right)
\end{equation}

Again the parameters to choose are the order of filter $c$, $M_1$, and the delay introduced $D$. This is because the order of $b$ can be chosen in such a way that all the postcursors are canceled by the feedback: $M_2=N_2+M_1-D-1$, and also the expression of the autocorrelation matrix significantly simplifies. The choice is carried out by selecting the values that minimize the functional $J_{min}$, this time being $M_1=9$ and $D=4$, and consequently $M2=16$ because $N_2=12$  In figure \ref{fig:C_c}, \ref{fig:C_psi} and \ref{fig:C_b} we plot the resulting filters $c$, $\psi$ at sampling time $T/2$ and $b$ at sampling time $T$. In the fractionally spaced version of the DFE the aim is again to obtain an overall pulse $\psi$ that satisfies the Nyquist conditions at time $T$ so the samples at even multiples of $T/2$ are not used in the equalizer and are therefore a degree of freedom of the system.

\begin{figure}[ht]
	\begin{center}   
		\includegraphics[width=\textwidth]{figs/C_c.png} 
		\caption{Magnitude of the impulse response of the filter $c$ (feedforward filter) for the receiver in point C.}
		\label{fig:C_c}
	\end{center}
\end{figure}

\begin{figure}[ht]
	\begin{center}   
		\includegraphics[width=\textwidth]{figs/C_psi.png} 
		\caption{Magnitude of the impulse response of the system $\psi$ in point C.}
		\label{fig:C_psi}
	\end{center}
\end{figure}

\begin{figure}[ht]
	\begin{center}   
		\includegraphics[width=\textwidth]{figs/C_b.png} 
		\caption{Magnitude of the impulse response of the filter $b$ (feedback filter) in point C.}
		\label{fig:C_b}
	\end{center}
\end{figure}

\section*{Point D} 









\begin{figure}[ht]
	\begin{center}   
		\includegraphics[width=\textwidth]{figs/D_c.png} 
		\caption{Magnitude of the impulse response of the filter $c$ (feedforward filter) for the receiver in point D.}
		\label{fig:D_c}
	\end{center}
\end{figure}

\begin{figure}[ht]
	\begin{center}   
		\includegraphics[width=\textwidth]{figs/D_psi.png} 
		\caption{Magnitude of the impulse response of the system $\psi$ in point D.}
		\label{fig:D_psi}
	\end{center}
\end{figure}

\begin{figure}[ht]
	\begin{center}   
		\includegraphics[width=\textwidth]{figs/D_b.png} 
		\caption{Magnitude of the impulse response of the filter $b$ (feedback filter) in point D.}
		\label{fig:D_b}
	\end{center}
\end{figure}

The following table sums up the final parameters for the 4 configurations A, B, C, D.

\begin{table}[htbp]
	\begin{center}
		\begin{tabular}{p{2.7cm}cccccc}
			\toprule
			& \multicolumn{2}{c}{Length of $h$} & \multicolumn{4}{c}{Parameters} \\
			\cmidrule(lr){2-3}
			\cmidrule(lr){4-7}
			Point & $N_1$ & $N_2$ & $M_1$ & $M_2$ & $D$ & $\bar{t}_0$ \\
			\midrule
			A &  4  &  4  & 5 & 0  & 2 & 17 \\
			B &  4  &  4  & 5 & 4  & 4 & 17 \\
			C & 10  &  12 & 9 & 16 & 4 & 21 \\
			D & 10  &  12 & 9 & 16 & 4 & 21 \\
			\bottomrule
		\end{tabular}
	\end{center}
	\label{tab:sumup}
	\caption{Choices for the various parameters in the different receiver implementations with respect to the length of $h$.}
\end{table} 

\section*{Point E}

In this point we use the Viterbi algorithm (VA) to implement the Maximum Likelihood criterion for data detection. 
For the execution of the algorithm we consider the signal $y_k$ at the output of filter $c$ in the schema at point B:
\begin{equation}
y_k = \psi_D a_{k-D} + \psi_{D+1} a_{k-D-1} + \dots +\psi_{D+M_2} a_{k-D-M_2} + w_k
\end{equation}
where $w_k$ includes the residual ISI and the noise. The useful signal $u_k$ can then be taken as $u_k=\psi_D a_{k-D} + \psi_{D+1} a_{k-D-1} + \dots +\psi_{D+M_2} a_{k-D-M_2}$ and if we call $y_k=z_k$ we have the usual setup for VA: $z_k = u_k+w_k$. We will have then that the number of states is $N_s=M^{M_2}$ where $M$ is 4, the cardinality of the QPSK constellation. In the specific case of point B $M_2=4$ so $N_s=256$. We define the state basing ourselves of the choice of $u_k$ so:
\begin{equation}
\begin{split}
\vt{s}_k = \{a_{k-D}, a_{k-D-1}, a_{k-D-2}, \dots, a_{k-(M_2-1)}\} = \\
= \{a_{k-D}, a_{k-D-1}, a_{k-D-2}, a_{k-D-3}\}
\end{split}
\end{equation}

\section*{Point F}

\section*{Simulation results}



\begin{figure}[ht]
	\begin{center}   
		\includegraphics[width=\textwidth]{figs/SNR_Pe.png} 
		\caption{Results of the simulation over values of the SNR at the channel output from 8 dB to 14 dB.}
		\label{fig:SNR}
	\end{center}
\end{figure}


 




\begin{thebibliography}{15}
	
	\bibitem{nevio<3}
	Nevio Benvenuto, Giovanni Cherubini,
	\textit{Algorithms for Communication Systems and their Applications}. 
	Wiley, 2002.
	

	
\end{thebibliography}

\end{document}