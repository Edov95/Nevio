\documentclass[a4paper,11.5pt]{article}
\usepackage[latin1]{inputenc}
\usepackage[T1]{fontenc}
\usepackage[english]{babel}
\usepackage{graphicx}
\usepackage{amsmath}
\usepackage{amsfonts}
\usepackage{multirow}
\usepackage{booktabs}
\usepackage{bbold}
\usepackage{mathtools}
\usepackage{mathrsfs}
\usepackage{enumitem}
\usepackage{array}
\usepackage{float}

\setlength{\parindent}{0pt}
\DeclarePairedDelimiter{\floor}{\lfloor}{\rfloor}
\DeclarePairedDelimiter{\ceil}{\lceil}{\rceil}

\newcommand{\vt}{\boldsymbol}

\title{Digital Communications - HW4}
\author{Jacopo Pegoraro, Edoardo Vanin}
\date{04/06/2018}

\begin{document}

\maketitle

We want to implement and evaluate the performances of two modulation schemes in the case of uncoded and coded bits. The first system we consider is a matched filter/DFE receiver with single carrier modulation. The information signal in this case is transmitted through the channel and equalized at the receiver by canceling the ISI due to the postcursors of the total impulse response with feedback, while the ISI due to the precursors is reduced by the feedforward filter $c$. In the case of OFDM instead we use a multi-carrier approach and we carry out the equalization by using the \emph{cyclic prefix} method. 

\section*{System setup}

The setup of the system is carried out differently for the uncoded and coded cases. The starting point is a sequence of bits $b_l$ on sampling time $T_{bit}$, produced by a PN sequence with parameter $r=20$ and length $L=2^r-1$ repeated once.

\subsection*{Uncoded}

In the uncoded case the sequence $b_l$ directly mapped into a sequence of QPSK symbols $a_k$ at sampling time $T_a=2T_{bit}$ through a bitmap function that uses Gray coding. At the receiver the detected sequence will pass through an inverse bitmap before the computation of the probability of bit error $P_{bit}$.

\subsection*{Coded}

In the coded case the sequence is first coded using an irregular LDPC code with rate $1/2$ and a codelength of $N = 64800$. This operation produces a sequence of bits $c_m$ with double the length of $b_l$ and sampling time $T_{cod}$. To improve the performances by reducing the effect of burst errors, $c_m$ is passed to an interleaver $43 \times 41$ in which we write the input bits by row and read them by column to perform a scramble.

At this point the resulting bit sequence $c_p'$ at $T_a = 2T_{cod}$ is transmitted using the single carrier and OFDM schemes. At the receiver, once the detection has taken place, the reverse operations are carrier out to derive the detected bit sequence $\hat{b}_l$ and compute $P_{bit}$. In particular a LDPC decoder and a deinterleaver $41\times 43$ are used.

\section*{Single Carrier with DFE}

%% Togliere un po' di merda

The system takes a sequence of input symbols $a_k$ at sampling time $T=1$ and applies an upsampling of factor 4, obtaining $a_k'$ at $T/4$. This new sequence is then filtered by $q_c$ as described by the following difference equation:
\begin{equation}\label{eq:q_c}
s_c(nT/4) = 0.67 s_c((n-1)T/4) + 0.7424 a_{n-5}
\end{equation}
After the filtering white noise is added. From the following relations we can derive $\sigma_w^2$, the variance of the complex valued Gaussian noise:
\begin{equation}
\Gamma = \frac{M_{s_c}}{N_0\frac{1}{T}} = \frac{\sigma_a^2 E_{q_c}}{\sigma_w^2} \longrightarrow \sigma_w^2 = \frac{\sigma_a^2 E_{q_c}}{\Gamma} = 2\sigma_I^2
\end{equation}
where $\sigma_I^2$ is the variance per component. In addition we can also compute the PSD as $N_0=\sigma_w^2 T_c=\sigma_w^2/4$, because the sampling time $T_c$ at which we add the noise is $T/4$.
In figure \ref{fig:qc} we plot the impulse response and the frequency response of the filter $q_c$.

\begin{figure}[H]
	\begin{center}   
		%\includegraphics[width=\textwidth]{figs/q_c.png} 
		%\includegraphics[width=\textwidth]{figs/Qc.png} 
		\caption{Impulse response of the filter $q_c$ at $T/4$.}
		\label{fig:qc}
	\end{center}
\end{figure} 

At the receiver we have a matched filter $g_{M}$ (see figure \ref{fig:A_gm}), obtained from $q_c$ as $g_M=q_c^*(t_0-t)$. For simplicity in the last formula we have denoted the filters as is they were defined on continuous time while in the actual simulation they are at $T/4$. 

\begin{figure}[H]
	\begin{center}   
		%\includegraphics[width=\textwidth]{figs/A_gm.png} 
		\caption{Impulse response of the matched filter $g_{M}$ for the receiver in point A.}
		\label{fig:A_gm}
	\end{center}
\end{figure} 

The output of the matched filter is then sampled at $T$ starting from an initial offset called \emph{timing phase} $t_0$. In our case the choice of $t_0$ is made easy by the presence of the matched filter, as we can just choose the value $\bar{t}_0$, multiple of $T/4$, that is the index of the peak of the correlation between $q_c$ and $g_M$, then $t_0$ will be equal to $\bar{t}_0 T/4$. Following this reasoning we chose $\bar{t}_0=16$ (17 with Matlab indexing), equal also to the index of the last sample of $g_M$ (see figure \ref{fig:A_gm}).

We equalize with a DFE, that is made of two filters called feedforward and feedback filter denoted by $c$ and $b$. The feedforward filter has the role of equalizing only the precursors of the overall impulse response, while the ISI due to postcursors will be canceled by filter $b$ positioned on a feedback loop between the output of the threshold detector and its input. 

\begin{figure}[H]
	\begin{center}   
		%\includegraphics[width=\textwidth]{figs/B_gm.png} 
		\caption{Impulse response of the matched filter $g_{M}$ for the receiver in point B.}
		\label{fig:B_gm}
	\end{center}
\end{figure}

The computation of the optimal filters $c$ and $b$ is carried out using the Wiener filter approach. The relation between the input random process and the output is:
\begin{equation}
\begin{split}
y_k &= x_{FF,k} + x_{FB,k} \\
&= \sum_{i=0}^{M_1-1}c_ix_{k-i} + \sum_{j=1}^{M_2}b_ja_{k-D-j}
\end{split} 
\end{equation}
where $M_1$ is the order of the feedforward filter, $M_2$ is the order of the feedback filter and $a_{k-D}$ are the already detected past symbols fed back through $b$. Defining postcursors and precursors as in point A, we have that we can apply the Wiener-Hopf equations on the process:
\begin{equation} \label{eq:yk}
y_k = \sum_{i=0}^{M_1-1}c_i \left(x_{k-i}-\sum_{j=1}^{M_2}h_{j+D-i}a_{k-j-D} \right)
\end{equation}

The result can be easily computed as $c_{opt} = \vt{R}^{-1}\vt{p}$ once we find the autocorrelation matrix $\vt{R}$ and the correlation vector $\vt{p}$, expressed as \cite{nevio<3}:

\begin{equation} \label{eq:wienerR}
\mathbf{[R]}_{p,q} = \sigma_a^2 \left( \sum_{j=-N_1}^{N_2}h_jh^*_{j-(p-q)}-\sum_{j=1}^{M_2}h_{j+D-q}h^*_{j+d-p} \right) + r_{\tilde{w}}(p-q)
\end{equation}
\begin{equation} \label{eq:wienerp}
\mathbf{[p]}_p = \sigma_a^2 h^*_{D-p} \quad\quad\quad\quad\quad\quad p = 0,1,\dots,M_1-1
\end{equation}

where for a QPSK scheme $\sigma_a^2=2$ because it is the sum of two orthogonal components each with power $1$. The values of $r_{\tilde{w}}$ are the result of the autocorrelation of the noise after being filtered by $g_M$, so being the noise white we have $r_{\tilde{w}}(n)=N_0r_{g_M}(nT)$. At this point we can define the overall impulse response up to the threshold detector $\psi = h*c_{opt}$ and derive the optimal coefficients for filter $b$ as $b_i=-\psi_{i+D}$ for $i=1,\dots,M_2$.

The value of the cost function $J_{min}$ obtained using these the optimal filters is :
\begin{equation} \label{eq:jmin}
J_{min} = \sigma^2_a \left( 1-\sum_{l=0}^{M_1-1} c_{opt,l}h_{D-l}\right)
\end{equation}

Again the parameters to choose are the order of filter $c$, $M_1$, and the delay introduced $D$. This is because the order of $b$ can be chosen in such a way that all the postcursors are canceled by the feedback: $M_2=N_2+M_1-D-1$, and also the expression of the autocorrelation matrix significantly simplifies. The choice is carried out by selecting the values that minimize the functional $J_{min}$, this time being $M_1=5$ and $D=4$, and consequently $M_2=4$ because $N_2=4$.Once we noticed that for a certain value of $D$ the functional could only decrease, we let $M_1$ vary to choose the optimal couple of parameters. The following figure shows the knee of the curve at $M_1=5$ with $D$ fixed at 4:

\begin{figure}[H]
	\begin{center}   
		%\includegraphics[width=\textwidth]{figs/B_Jmin.png} 
		\caption{Behavior of $J_{min}$ for $D=4$ varying $M_1$.}
		\label{fig:B_Jmin}
	\end{center}
\end{figure}


In figure \ref{fig:B_c}, \ref{fig:B_psi} and \ref{fig:B_b} we plot the resulting filters $c$, $\psi$ and $b$ at sampling time $T$. 

\begin{figure}[H]
	\begin{center}   
		%\includegraphics[width=\textwidth]{figs/B_c.png} 
		\caption{Magnitude of the impulse response of the filter $c$ (feedforward filter) for the receiver in point B.}
		\label{fig:B_c}
	\end{center}
\end{figure}

\begin{figure}[H]
	\begin{center}   
		%\includegraphics[width=\textwidth]{figs/B_psi.png} 
		\caption{Magnitude of the impulse response of the system $\psi$ in point B.}
		\label{fig:B_psi}
	\end{center}
\end{figure}

\begin{figure}[H]
	\begin{center}   
		%\includegraphics[width=\textwidth]{figs/B_b.png} 
		\caption{Magnitude of the impulse response of the filter $b$ (feedback filter) in point B.}
		\label{fig:B_b}
	\end{center}
\end{figure}



\section*{OFDM with cyclic prefix}

The system considered in this case is represented in figure \ref{fig:ofdm_schema}:

\begin{figure}[H]
	\begin{center}   
		\includegraphics[width=\textwidth]{figs/OFDM_schema.png} 
		\caption{Block diagram of the OFDM system with the cyclic prefix method.}
		\label{fig:ofdm_schema}
	\end{center}
\end{figure}

The input sequence of symbols $a_k$ is arranged in a matrix with $\mathcal{M}=512$ rows and on these blocks of length $\mathcal{M}$ we compute the IDFT to obtain the sequence $A_k[i]$ for $i=0,\dots , \mathcal{M}-1$. At this point we used the cyclic prefix method to perform equalization: a prefix of length $N_{px}$ taken from the end of the block obtained form IDFT is repeated on top of the same block. In this way, choosing $N_{px}\geq N_c-1$ (where $N_c$ is the length of the channel impulse response) the data of the next block will be separated from the present one by $N_c-1$ or more samples, that is enough to fill the channel and not have any ICI between consecutive blocks. At the receiver the additional $A_k[i]$ with $i=\mathcal{M}-1-N_{px},\dots ,\mathcal{M}-1$ can be just discarded.

\subsection*{OFDM channel}

The channel, represented as a single block in figure \ref{fig:ofdm_schema} contains all blocks depicted in figure \ref{fig:ofdm_channel_schema}

\begin{figure}[H]
	\begin{center}   
		\includegraphics[width=10cm]{figs/OFDM_channel_schema.png} 
		\caption{Channel + transmitter and receiver filters in OFDM.}
		\label{fig:ofdm_channel_schema}
	\end{center}
\end{figure}

where in our case $g_{tx}$ and $g_{Rc}$ are taken both as square root raised cosine filters, and the channel impulse response is the $q_c$ used also for the single carrier channel: 
\begin{equation}\label{eq:q_c}
s_c(nT/4) = 0.67 s_c((n-1)T/4) + 0.7424 a_{n-5}
\end{equation}
The system takes a sequence of input symbols $s_k$ at sampling time $T_{OFDM}=T_{block}/(M + N_{px})$ and applies an upsampling of factor 4, obtaining $a_k'$ at $T_c = T_{OFDM}/4$. This new sequence is then filtered by $g_{\sqrt{rcos}}$ where the impulse response is described by the following equation:

\begin{equation}\label{eq:g_rcos}
g_{\sqrt{rcos}}(n) = \frac{\sin\left[\pi\left(1 - \rho\right)\frac{n}{T_c}\right] + 4\rho\frac{n}{T_c}\cos\left[\pi\left(1 - \rho\right)\frac{n}{T_c}\right]}{\pi\left[1-\left(4\rho\frac{n}{T_c}\right)^2\right]\frac{n}{T_c}}
\end{equation}

After the convolution with the transmit filter, we send the data through the channel with the impulse response given by (\ref{eq:q_c}) and white noise is added. From the following relations we can derive $\sigma_w^2$, the variance of the complex valued Gaussian noise:
\begin{equation}
\Gamma = \frac{M_{s_c}}{N_0\frac{1}{T}} = \frac{\sigma_a^2 E_{g_{\sqrt{rcos}} * q_c}}{\sigma_w^2} \longrightarrow \sigma_w^2 = \frac{\sigma_a^2 E_{g_{\sqrt{rcos}} *q_c}}{\Gamma} = 2\sigma_I^2
\end{equation}

At the receiver we have the matched filter of the transmission filter which is another $g_{\sqrt{rcos}}$ with the impulse response given by (\ref{eq:g_rcos}).
The output of the matched filter is then sampled at $T_{OFDM}$ starting from an initial offset called \emph{timing phase} $t_0$. We can just choose the value $\bar{t}_0$, multiple of $T_{OFDM}/4$ and we have chosen $\bar{t}_0=90$ (91 with Matlab indexing) "because this is the value with the best performances"



\begin{figure}[H]
	\begin{center}   
		\includegraphics[width=\textwidth]{figs/Pbit_uncoded.png} 
		\caption{$P_{bit}$ obtained for various values of the SNR with DFE equalization and OFDM, no channel coding used.}
		\label{fig:Pbit_uncoded}
	\end{center}
\end{figure}

\begin{figure}[H]
	\begin{center}   
		\includegraphics[width=\textwidth]{figs/Pbit_coded.png} 
		\caption{$P_{bit}$ obtained for various values of the SNR with DFE equalization and OFDM, LDPC coding has been used.}
		\label{fig:Pbit_coded}
	\end{center}
\end{figure}
 
\begin{thebibliography}{15}	
	\bibitem{nevio<3}
	Nevio Benvenuto, Giovanni Cherubini,
	\textit{Algorithms for Communication Systems and their Applications}. 
	Wiley, 2002.
\end{thebibliography}

\end{document}